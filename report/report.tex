\documentclass{article}

\usepackage{verbatim}

\newcommand{\symb}{{symbols}}
\newcommand{\signature}{{sign}}

\begin{document}

    This report is, at the moment, a draft reminding us the different steps we
    considered (even trivially false ones) to reach our conclusions.

   \section{Deciding version of a library used by a program.} 

   \subsection{First try to define the problem.}

   Let $\mathcal{D}$ a database of versionned libs to be set of $n$ tuples such that:
    \[ \mathcal{D} = \{ (n_i, v_i, s_i, \mathcal{B}_i) \}_{\{i \in \{1 \dots n\}\}}  \]
    where $n_i$ is the library name, $v_i$ a the library version, $s_i$ the source
    code of the library, and $\mathcal{B}_i$ is the set of binary compiled
    from the source code $s_i$. 

    Let $L$ be a binary code of a library.

    We want to find a function $V(L,D)$ such that if $L$ is compiled from a
    source code $C$ and $(N, V, C, \_) \in \mathcal{D}$ then $V(L,D)$ returns
    $(N,V)$.

    \subsection{Manual methods to look for a library version.}
    Finding some version numbers inside the binary file as data.
    It works for openssl but not with zlib.
    Issue, it does not seem to be a standard practice.

    Looking for symbols in the file. 
    Hypothesis: It returns the calls to the libs.
    Issue, it does not really help to get the version.

    Looking for version info function in the symbols of the file. Would
    probably work for zlib if we understood how this function works.
    Example: In the Good..apk, a library calls or includes the function zlibVersion statically. This function returns the version number, but we were not able to retrieve it in the binary code of the function.
    
    Semantically checking last pushed versions of functions seems to be an
    interesting approach, we may have a look at it later.

    \subsection{Naive approach}
    We could simply try if there exists $(N, V, \_, \{\dots, B, \dots\} \in
    \mathcal{D}$ such that $L = B$.

    But after some tests and look on the \verb|diff| of these binaries we
    concluded that major differences can come from the compiling options,
    compilers and related stuff.

    Question: can we generate all possible compiled binaries in our database ? 
    
    Answer: It would certainly be a bad idea\dots

    \subsection{Second approach}
    Can we find a smarter way to compute $V(L,D)$ ?

    We could find a function $\signature$ such that $\signature(L) =
    \signature(B)$.

    We plan to reuse a signature function designed for malware.

    Expected issues: since a new version is not that different from the former
    one, we do not know if we will obtain good results.

\end{document}
